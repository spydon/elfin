\documentclass[a4paper,12pt]{article}
\usepackage{fullpage}
\usepackage[T1]{fontenc}
\usepackage{amsmath}
\usepackage{amssymb}
\usepackage[utf8]{inputenc}
\usepackage{color}
\usepackage{authblk}
\usepackage{todonotes}
\usepackage{caption}
\usepackage{url}
\usepackage{float}
\usepackage{sectsty}
\usepackage{pdfpages}
\usepackage[section]{placeins}
\DeclareCaptionFont{white}{\color{white}}
\DeclareCaptionFormat{listing}{\colorbox{gray}{\parbox{\textwidth}{#1#2#3}}}
\captionsetup[lstlisting]{format=listing,labelfont=white,textfont=white}

\usepackage{setspace}
\usepackage[toc,page]{appendix}
\usepackage{framed}
\usepackage{geometry}

\usepackage{alltt}
\usepackage{subfig}

% Change section fonts
\allsectionsfont{\sffamily}

% For code box
\usepackage{xcolor}
\usepackage{listings}
\usepackage{caption}
\DeclareCaptionFont{white}{\color{white}}
\DeclareCaptionFormat{listing}{%
  \parbox{\textwidth}{\colorbox{gray}{\parbox{\textwidth}{#1#2#3}}\vskip-4pt}}
  \captionsetup[lstlisting]{format=listing,labelfont=white,textfont=white}
  \lstset{frame=lrb,xleftmargin=\fboxsep,xrightmargin=-\fboxsep}
% End code box

\usepackage{cite}

% General parameters, for ALL pages:
\renewcommand{\topfraction}{0.9}	% max fraction of floats at top
\renewcommand{\bottomfraction}{0.8}	% max fraction of floats at bottom
% Parameters for TEXT pages (not float pages):
\setcounter{topnumber}{2}
\setcounter{bottomnumber}{2}
\setcounter{totalnumber}{4} % 2 may work better
\setcounter{dbltopnumber}{2} % for 2-column pages

\addtolength{\topmargin}{0.5in}

\usepackage{fancyvrb}

\usepackage{tikz} \usetikzlibrary{trees}
\usepackage{hyperref} % should always be the last package

% useful colours (use sparingly!):
\newcommand{\blue}[1]{{\color{blue}#1}}
\newcommand{\green}[1]{{\color{green}#1}}
\newcommand{\red}[1]{{\color{red}#1}}

% useful wrappers for algorithmic/Python notation:
\newcommand{\length}[1]{\text{len}(#1)}
\newcommand{\twodots}{\mathinner{\ldotp\ldotp}} % taken from clrscode3e.sty
\newcommand{\Oh}[1]{\mathcal{O}\left(#1\right)}

% useful (wrappers for) math symbols:
\newcommand{\Cardinality}[1]{\left\lvert#1\right\rvert}
\newcommand{\Ceiling}[1]{\left\lceil#1\right\rceil}
\newcommand{\Floor}[1]{\left\lfloor#1\right\rfloor}
\newcommand{\Iff}{\Leftrightarrow}
\newcommand{\Implies}{\Rightarrow}
\newcommand{\Intersect}{\cap}
\newcommand{\Sequence}[1]{\left[#1\right]}
\newcommand{\Set}[1]{\left\{#1\right\}}
\newcommand{\SetComp}[2]{\Set{#1\SuchThat#2}}
\newcommand{\SuchThat}{\mid}
\newcommand{\Tuple}[1]{\langle#1\rangle}
\newcommand{\Union}{\cup}
\usetikzlibrary{positioning,shapes,shadows,arrows}

\renewcommand{\abstractname}{Elfin - URL Shortener}

\title{\textbf{Elfin - URL Shortener}}
\author{Lukas Klingsbo}
\begin{document}
\maketitle

\begin{abstract}
I was given this task by Uprise, as an exercise to show my capabilities.\\
This URL shortener is built with Scala 2.11.7 and the Lift 2.6 Framework.

\textbf{Disclaimer:} This is the first Lift project that I have 
coded so please forgive any horrible breaking of convention 
that I might have unintentionally made. 
\begin{center}
If you have any questions,\\
feel free to contact me on:\\
\vspace{10pt}
Lukas Klingsbo\\
+46737-42 43 45\\
lukas.klingsbo@gmail.com\\
\end{center}

\section{Background}
The task was to make a scalable URL shortener that deterministically shortened 
the URL's fed to it.

\section{Design decisions}
The shortener core code was written with the help of MurmurHash3 which I use 
to convert the $URL \Rightarrow Int \Rightarrow String \Rightarrow URL$. 

Another design choice that was considered was using an auto incremented value 
in a database that was to be converted to a string. This would have yielded 
shorter strings and avoided the problem with collisions that occur when using 
short hashes. The first design choice was chosen as the exercise stated that 
the same link preferably always should produce the same short URL. It also 
gives the advantage that the keys does not have to be synchronized on creation 
to the distributed KV-store, they still have to be synchronized for fetching.

\pagenumbering{gobble} % Remove page numbers (and reset to 1)
\end{abstract}
\end{document}
